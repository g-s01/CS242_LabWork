% declaring the document type
\documentclass{article}
% declaring the encoding 
\usepackage[utf8]{inputenc}
% for enhancing math-formulas
\usepackage{amsmath}
% for creating hyperlinks
\usepackage{hyperref}
% for writing multiline comments without %
\usepackage{comment}
% for coloring the output text
\usepackage{xcolor}

\title{Hello World!}
\author{Gautam Sharma}
\date{January 1, 1831}
% to suppress the page number in the document
\pagenumbering{gobble}
\begin{document}

\maketitle

\section{Getting Started}
$\textbf{Hello World!}$ Today I am learning $\LaTeX$. $\LaTeX$ is a great program for writing math. I can write in-line math such as $a^2+b^2=c^2$. I can also give equations their own space:
\begin{equation}
    \gamma^2+\theta^2=\omega^2
\end{equation}
``Maxwell's equations'' are named for James Clark Maxwell and are as follow:
\begin{flalign}
     &  & \vec{\nabla} \cdot \vec{E}  & = \frac{\rho}{\epsilon_0} &  & \text{Gauss's Law}\label{eq:a’} \\
     &  & \vec{\nabla} \cdot \vec{B}  & = 0  &  &  \text{Gauss' Law for Magnetism}\label{eq:b’} \\
     &  & \vec{\nabla} \times \vec{E}  & = -\frac{\partial \vec{B}}{\partial t} &  & \text{Faraday's Law of Induction}\label{eq:c’} \\
     &  & \vec{\nabla} \times \vec{B}  & = \mu_0 \left(\epsilon_0 \frac{\partial \vec{E}}{\partial t} + \vec{J}\right)  &  &  \text{Ampere's Circuital Law}\label{eq:d’}
\end{flalign}
% using xcolor package to color the equation numbers blue, linking them to the corresponding equation number by using the label for each equation and the hyperref package
Equations \textcolor{blue}{\ref{eq:a’}, \ref{eq:b’}, \ref{eq:c’}} and \textcolor{blue}{\ref{eq:d’}} are some of the most important in Physics.

\section{What about Matrix Equations?}
% the matrix [a_(ij)] where 1 <= i, j <= n
$
 \begin{pmatrix}
  a_{11} & a_{12} & \cdots & a_{1n}\\ 
  a_{21} & a_{22} & \cdots & a_{2n}\\
  \vdots & \vdots & \ddots & \vdots\\
  a_{n1} & a_{n2} & \cdots & a_{nn}
\end{pmatrix}
$
% the matrix [v_1, v_2, ..., v_n]^T (where ^T = taking transpose)
$
\begin{bmatrix}
v_1\\
v_2\\
\vdots\\
v_n
\end{bmatrix}
$
=
% the matrix [w_1, w_2, ..., w_n]^T (where ^T = taking transpose)
$
\begin{matrix}
w_1\\
w_2\\
\vdots\\
w_n
\end{matrix}
$
\end{document}
